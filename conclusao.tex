A massificação do uso das redes sociais gera uma crescente produção de dados.
Entretanto, as informações sendo geradas são dispostas de forma não estruturada, dificultando sua extração.
Dado o volume de dados produzidos, se torna cada vez mais custoso realizar este processo manualmente.
Portanto, é fundamental o desenvolvimento de técnicas de processamento de linguagem natural capazes de auxiliar neste
procedimento.

O presente trabalho teve como objetivo o desenvolvimento de um método de gerar classificadores de análise de sentimento
para redes sociais sem dependência da anotação manual de bases de dados.
Classificadores formados por algoritmos de aprendizado de máquina vem obtendo bons resultados na mineração de opinião.
Todavia, estes dependem de dados de treinamento, os quais sua formação é custosa visto que sua formação depende da
anotação manual dos casos, dificultando, por exemplo, a reprodução destes classificadores para diferentes idiomas ou
redes sociais.

Com o desenvolvimento de classificadores cada vez mais complexos, como os englobados pelas técnicas de
\textit{deep learning}, cresce a dependência de grandes volumes de dados de treinamento.
O método da anotação ruidosa por supervisão distante, apesar de apresentar dificuldades decorrentes da seleção das
características correlacionada com as classes, se mostrou eficiente na formação de bases de treinamento.

Observou-se que embora algoritmos de aprendizado de máquina tradicionalmente aplicadas ao processamento de linguagem
natural atinjam resultados positivos nesta tarefa, a utilização de técnicas de \textit{deep learning} é capaz elevar o
nível de desempenho obtido.

Portanto, foi desenvolveu um método de classificação que não depende de anotação manual de dados de treinamento.
Isto foi possível pois as representações de texto utilizadas são obtidas através de algoritmos de aprendizado de máquina
não supervisionados e as redes neurais convolucionais são treinadas a partir de dados anotados por supervisão distante.

\section{Trabalhos Futuros}

% possibilidade de treinar o w2v
% neutro
% lstm
% char
