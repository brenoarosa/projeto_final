Este capítulo visa descrever a elaboração de um modelo de \textit{deep learning} capaz de classificar sentimento,
positivo ou negativo, de \textit{tweets}.
A produção desse modelo será feita a partir de base de dados anotada de maneira automatizada, permitindo a sua
fácil replicação.

\section{Bases de Dados}

O trabalho de Go \textit{et at.}~\cite{go09} visa definir um sistema de classificação de sentimento sem a necessidade de
anotação manual dos dados para treinamento, utilizando a técnica de supervisão distante.
A não dependência de anotação dos dados e a facilidade de coleta dos \textit{tweets}, por meio de interface programável,
viabilizam a formação de uma grande base de treinamento.
Para a realização deste trabalho foram coletados cerca de 40 milhões de mensagens do Twitter filtrando-as apenas pelo
idioma inglês.
Posteriormente, se aplica o método de classificação ruidosa de sentimento desenvolvido por Read~\cite{read05} como
especificado na seção~\ref{sec:distant_supervision} nos \textit{tweets} coletados.
Nesse método, são definidos grupos de \textit{emoticons} positivos e negativos.
Mensagens que possuírem \textit{emoticons} pertencentes a algum destes grupos serão anotadas com as respectivas classes,
caso \textit{emoticons} de ambos os grupos estejam presentes em uma mesma mensagem esta será descartada.
Os \textit{emoticons} utilizados para anotação da base de dados foram removidos das mensagens para evitar introduzir
viés ao classificador.
A tabela~\ref{tab:emoticons} mostra alguns dos emoticons escolhidos para compor a anotação ruidosa.

\begin{table}[h]
    \begin{center}
        \begin{tabular}{| c | c |}
        \hline
        \textbf{Negativos} & \textbf{Positivos} \\ \hline
        :( & :) \\ \hline
        =( & =) \\ \hline
        :-( & :-) \\ \hline
        :'( & :D \\ \hline
        \end{tabular}
        \caption{Exemplos de emoticons selecionados para aplicação de supervisão distante.}
        \label{tab:emoticons}
    \end{center}
\end{table}

A base de teste por sua vez é composta de \textit{tweets} anotados manualmente.
Essa base é formada pela coletânea de dados disponibilizados pela conferencia anual \textit{Semantic Evaluation}
(SemEval)~\cite{semeval17} entre os anos de 2013 a 2017.
Esta base é composta de cerca de 70 mil \textit{tweets} e processo de anotação destes dados foi feito manualmente
através da plataforma CrowdFlower, mais detalhes sobre a formação da base de dados são apresentados por Rosenthal
\textit{et al.}~\cite{rosenthal17}, organizadores do evento.
Novamente foram removidos os \textit{emoticons} presentes na anotação ruidosa.

\section{Métricas} \label{sec:metrics}

Uma vez definidos os dados e as técnicas de classificação, se faz necessário medir a qualidade dos modelos obtidos.
Esta seção apresentará as métricas utilizadas para seleção de parâmetros e avaliação dos algoritmos.

A forma mais direta de se medir a performance de um modelo é considerar a sua acurácia.
Todavia, a presença de desbalanceamento de classes nos dados pode favorecer a acertividade da classe mais presente
enquanto perde eficiencia na classificação da classe com menos exemplos.

% ROC - dissertacao fernando
A curva ROC (\textit{receiver operating characteristic})~\cite{bradley97} é uma forma de visualizar a performance de um
classificador binário para seleção de um ponto de operação.
Nela se apresenta a variação da probabilidade de detecção e de falso alarme variam com a escolha do patamar de decisão.
A área sob a curva pode ser utilizada como métrica de eficiência para um classificador.

% SP - dissertacao fernando
O índice SP é outro método de avaliação de classificadores binários.
A equação~\ref{eq:sp} descreve a formação desse índice na qual $P_c$ representa a probabilidade de detecção da classe
positiva enquanto $P_{nc}$ caracteriza a probabilidade de não obtenção de falso alarme~\cite{ciodaro12}.

\begin{equation} \label{eq:sp}
    SP = \sqrt{\sqrt{P_c P_{nc}}(\frac{P_c + P_{nc}}{2})}
\end{equation}

O patamar de decisão do classificador será escolhido de maneira a maximizar o índice SP.

% validaçao cruzada, ver dissertação do junior
Uma maneira de garantir que os resultados obtidos pelas métricas apresentadas sejam consistentes independente do
conjunto de dados na qual será aplicado é utilizar a validação cruzada.
Neste trabalho será utilizado o método de validação cruzada de K-partições.
Este processo~\cite{kohavi95} consiste na separação aleatória da base de dados em $k$ conjuntos mutualmente exclusivos
de tamanhos similares.
Posteriormente são executadas $k$ rodadas, na qual o conjunto correspondente a cada rodada será utilizado para validação
do modelo enquanto os outros $k-1$ conjuntos servirão para seu treinamento.
O treinamento pode ser considerado consistente se obtiver pouca variação nos resultados das $k$ rodadas.

\section{Desenvolvimento} \label{sec:desenvolvimento}

% Etapa 1
O presente trabalho será dividido em três etapas.
Na primeira etapa utilizou-se a base de dados provida por Go \textit{et al.}~\cite{go09} replicando as técnicas
abordadas em seu artigo visando validar os pré-processamentos e algoritmos aplicados.
Para tal, serão utilizadas as bases de dados tanto de treinamento como de testes disponibilizadas por Go
\textit{et at.}~\cite{go09}.

Começou-se tokenizando os \textit{tweets}, durante esse processo removeram-se stopwords; links; referências a usuários;
e cada token foi transformado para forma minuscula.
Replicou-se como entrada do algoritmo de aprendizado apenas a representação dos tokens por unigrama, essa escolha foi
feita por ser a representação mais simples e por Go \textit{et at.}~\cite{go09} mostrar que há pouca variação de
resultado entre as diferentes representações.

Finalmente, foram treinados os algoritmos de Naïve Bayes e SVM.
Utilizou-se Naïve Bayes com distribuição multinomial, por se adequar ao pré processamento utilizado.
Variou-se o parâmetro de suavização de distribuição de laplace, de maneira a otimizar a área da curva ROC.
Para treinamento do modelo por \textit{support vector machine}, dada a grande quantidade de dados e as limitações
computacionais,foi empregado o treinamento a partir do método do gradiente como descrito por Suykens e
Vandewalle~\cite{suykens99}.
A função \textit{kernel} utilizada foi linear.
Neste caso, variou-se parâmetro de regularização L2 também visando maximizar a área da curva ROC.
Empregou-se validação cruzada por K-partições, com 5 partições, no treinamento dos dois algoritmos.

% Etapa 2
A segunda etapa consiste em treinar os mesmos algoritmos de Naive Bayes e SVM utilizados anteriormente porém utilizando
a base de dados anotada com supervisão distante.
Os resultados obtidos nesta fase servirão como base de comparação entre os modelos lineares e modelos formados técnicas
de \textit{deep learning}.
Adicionamente, será validado o processo de captação de dados.
Nesta fase replicou-se os mesmos procedimentos práticos da etapa anterior.

% Etapa 3
Por fim, a terceira e última etapada é formada pela aplicação de redes convolucionais em textos como descrito por
Kim~\cite{kim14}.
Nesta fase será medido o impacto da utilização desta técnica na classificação de mensagens.
Serão variados diferentes pré-processamentos e parametros das redes para analisar sua influência na eficiência do
modelo.

Para utilização de modelos de redes convolucionais, obtém-se primeiro uma representação dos dados em um
\textit{embedding}.
Foram avaliados duas opções de \textit{embeddings}: a utilização de um \textit{embedding} Word2Vec pré-treinado a
partir de notícias, disponibilizados pelo Google; e o treinamento de um \textit{embedding} Word2Vec a partir da base de
treinamento composta por \textit{tweets}.
Este experimento visa verificar as implicações de utilização de representações obtidas a partir de textos da mesma
fonte a ser classificada.
O número de dimensões do Word2Vec treinado a partir de \textit{tweets} foi baseados no modelo disponibilizado pelo
Google, com 300 dimensões.

Geraram-se modelos de redes convolucionais aplicando como entrada a representação obtida pelo Word2Vec.
A escolha dos hiperparâmetros a serem testados foi baseada na submissão ao SemEval de Derius
\textit{et al.}~\cite{deriu16}, a qual obteve o melhor resultado da SemEval 2016.
Compões os hiperparâmetros variados: número de camadas, número de filtros convolutivos por camada, tamanho dos filtros,
tamanho do \textit{pooling}.
Começou-se com redes com poucos parâmetros a serem ajustados e acompanhou-se o efeito de aumentar a complexidade da
rede.
Uma vez atingido um modelo no qual aparece a presença de \textit{overfitting}, aplicou-se as técnicas de regularização L2
e \textit{dropout}, com hiperparâmetros propostos por Derius \textit{et al.}
Enfim, continuou-se aumentando o número de neurônios da rede regularizada até atingir o ponto com valor máximo de área
sob a curva ROC.

Durante o treinamento das redes neurais convolucionais não foi utilizada a validação cruzada visto o alto custo
computacional de treinamento de cada rede.

% Foram testados tanto modelos com aplicação de \textit{fine-tuning} no \textit{embedding} quanto modelos em que os pesos do Word2Vec mantinham-se estáticos durante o treinamento.
