% Declaracao
\begin{center}
Declaração de Autoria e de Direitos
\end{center}

\vspace{0.5cm}

Eu, \emph{Breno Vieira Arosa} CPF \emph{131.187.117-95}, autor da monografia \emph{Análise de Sentimento em Textos Curtos Baseada em Processamento de Linguagem Natural e Aprendizado de Máquina}, subscrevo para os devidos fins, as seguintes informações:\\
1. O autor declara que o trabalho apresentado na disciplina de Projeto de Graduação da Escola Politécnica da UFRJ é de sua autoria, sendo original em forma e conteúdo.\\
2. Excetuam-se do item 1. eventuais transcrições de texto, figuras, tabelas, conceitos e ideias, que identifiquem claramente a fonte original, explicitando as autorizações obtidas dos respectivos proprietários, quando necessárias.\\
3. O autor permite que a UFRJ, por um prazo indeterminado, efetue em qualquer mídia de divulgação, a publicação do trabalho acadêmico em sua totalidade, ou em parte. Essa autorização não envolve ônus de qualquer natureza à UFRJ, ou aos seus representantes.\\
4. O autor pode, excepcionalmente, encaminhar à Comissão de Projeto de Graduação, a não divulgação do material, por um prazo máximo de 01 (um) ano, improrrogável, a contar da data de defesa, desde que o pedido seja justificado, e solicitado antecipadamente, por escrito, à Congregação da Escola Politécnica.\\
5. O autor declara, ainda, ter a capacidade jurídica para a prática do presente ato, assim como ter conhecimento do teor da presente Declaração, estando ciente das sanções e punições legais, no que tange a cópia parcial, ou total, de obra intelectual, o que se configura como violação do direito autoral previsto no Código Penal Brasileiro no art.184 e art.299, bem como na Lei 9.610.\\
6. O autor é o único responsável pelo conteúdo apresentado nos trabalhos acadêmicos publicados, não cabendo à UFRJ, aos seus representantes,  ou ao(s) orientador(es), qualquer responsabilização/ indenização nesse sentido.\\
7. Por ser verdade, firmo a presente declaração.\\

      \vspace{0.5cm}
      \begin{flushright}
         \parbox{10cm}{
            \hrulefill

            \vspace{-.375cm}
            \centering{Breno Vieira Arosa}

            \vspace{0.1cm}
         }
      \end{flushright}

\pagebreak

% Copyright
\vspace{0.5cm}

UNIVERSIDADE FEDERAL DO RIO DE JANEIRO \\
Escola Politécnica - Departamento de Eletrônica e de Computação \\
Centro de Tecnologia, bloco H, sala H-217, Cidade Universitária \\
Rio de Janeiro - RJ      CEP 21949-900\\
\vspace{0.5cm}

Este exemplar é de propriedade da Universidade Federal do Rio de Janeiro, que poderá incluí-lo em base de dados, armazenar em computador, microfilmar ou adotar qualquer forma de arquivamento.

É permitida a menção, reprodução parcial ou integral e a transmissão entre bibliotecas deste trabalho, sem modificação de seu texto, em qualquer meio que esteja ou venha a ser fixado, para pesquisa acadêmica, comentários e citações, desde que sem finalidade comercial e que seja feita a referência bibliográfica completa.

Os conceitos expressos neste trabalho são de responsabilidade do(s) autor(es).

\pagebreak

% Agradecimento
\begin{center}
\textbf{AGRADECIMENTO}
\end{center}
\vspace{0.5cm}

Agradeço, primeiramente, à minha família pelo suporte.

À meus amigos, que dão sentido a todas as jornadas.

Aos colegas do CERN e da TWIST, com quem muito evoluí.

À todos funcionários e professores que proporcionaram minha formação.

\pagebreak

% Resumo
\begin{center}
\textbf{RESUMO}
\end{center}
\vspace{0.5cm}

As redes sociais modificaram a forma como as pessoas interagem e se tornaram cada vez mais presentes em suas vidas.
A produção de conteúdo digital, por sua vez, acompanha este crescimento.
Este grande volume de dados produzido dificulta o processo de extração de informações, uma vez que tais dados são
majoritariamente não estruturados.

O campo de processamento de linguagem natural nos dá ferramentas para de auxiliar a automatização desse procedimento.
Dentre estas técnicas, algoritmos de aprendizado de máquina se mostraram eficientes classificadores de texto em tarefas
como a análise de sentimento.
Em paralelo, observou-se nos últimos anos o aparecimento de técnicas de \textit{Deep Learning} que romperam barreiras
de desempenho nas mais diversas áreas da inteligência artificial.
Porém, a eficiência destes modelos depende de grandes bases de dados de treinamento, as quais têm processo de formação
custosas visto que a anotação destes dados é feita manualmente.

Este trabalho apresenta a elaboração de um método para gerar classificadores de \textit{Deep Learning} para análise de
sentimento de mensagens de redes sociais, sem a necessidade de bases de dados anotadas manualmente.
Para tal, serão formadas bases de dados com anotação ruidosa que servirão para treinamento de redes neurais
convolucionais.
Serão avaliados os resultados obtidos pelos classificadores de \textit{Deep Learning} em comparação com algoritmos de
aprendizado de máquina tradicionalmente aplicados no processamento de linguagem natural.

\vspace{1.0cm}

\noindent Palavras-Chave: Aprendizado de Máquina, Deep Learning, Processamento de Linguagem Natural, Análise de Sentimento.

\pagebreak

% Abstract
\begin{center}
\textbf{ABSTRACT}
\end{center}
\vspace{0.5cm}

Social networks have changed the way people interact and they become more and more present in their lives.
The production of digital content, in turn, accompanies this growth.
This large amount of data produced hinders the process of extracting information, since such data are mostly
unstructured.

Tools like natural language processing are able to aid automating this procedure.
Among these techniques, machine learning algorithms have been shown to be efficient text classifiers in tasks such as
sentiment analysis.
In parallel, it has been observed in recent years the emergence of Deep Learning techniques that have broken performance
barriers in the most diverse areas of artificial intelligence.
However, the efficiency of these models depends on large training datasets, which have a costly production process since
the labelling of this data is done manually.

This work presents the elaboration of a method for generating Deep Learning classifiers for sentiment analysis of social
networks messages without the necessity of manually annotated datasets.
In this respect, a dataset will be formed with noisy annotation and will be used to train convolutional neural networks.
The results obtained by the Deep Learning classifiers will be evaluated in comparison to machine learning algorithms
traditionally applied in natural language processing.

\vspace{1.0cm}

\noindent Keywords: Machine Learning, Deep Learning, Natural Language Processing, Sentiment Analysis.

\pagebreak

% Siglas
\begin{center}
\textbf{SIGLAS}
\end{center}
\vspace{0.5cm}

AdaGrad - Adaptative Gradient

Adam - \textit{adaptative moment estimation}

AUC - \textit{area under the curve}

CBOW - Continuos Bag-of-Words

CNN - redes neurais convolucionais

EMQ - Erro Médio Quadrático

GAN - redes neurais geradoras adversárias

LSTM - Long Short Term Memory

MLP - \textit{multilayer perceptron}

NB - Naïve Bayes

ReLU - \textit{rectified linear unit}

RNN - redes neurais recorrentees

ROC - \textit{receiver operating characteristic}

SemEval - \textit{Semantic Evaluation}

SVM - \textit{Suport Vector Machine}

TF-IDF - \textit{term frequence-inverse document frequence}

UFRJ - Universidade Federal do Rio de Janeiro

W2V - Word2Vec

\pagebreak
