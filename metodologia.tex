Este capítulo visa descrever a elaboração de um modelo de \textit{deep learning} capaz de classificar sentimento,
positivo ou negativo, de \textit{tweets}.
A produção desse modelo será feita a partir de base de dados anotada de maneira automatizada, permitindo a sua
replicação sem dependência de anotação manual.

O presente trabalho será dividido em duas etapas.
A primeira se constitui em criar uma base de dados de treinamento de maneira como descrito por Go
\textit{et at.}~\cite{go09} e replicar os resultados obtidos com técnicas tradicionais de aprendizado de máquina
aplicados a texto, como apresentadas pelo autor.
Em seguida, reproduzir a aplicação de redes convolucionais em textos como descrito por Kim~\cite{kim14} e analisar o
impacto da utilização desta técnica na classificação de mensagens.

% remover ambos
O trabalho de Go \textit{et at.}~\cite{go09} visa definir um sistema de classificação de sentimento sem a necessidade de
anotação manual dos dados para treinamento, utilizando a técnica de supervisão distante.
Apesar do termo \textit{análise de sentimento} ter definição ambígua na literatura, nesse contexto estaremos tratando da
extração de polaridade, negativa ou positiva, de uma mensagem.
Em especial, as mensagens tratadas nesse trabalho são originárias do Twitter, serviço de microblogs.

Embora já existisse uma ampla gama de estudos em análise se sentimento, Go \textit{et at.} apresentaram o primeiro
trabalho referente a \textit{tweets}.
A principal característica das mensagens de microblogs são seu tamanho reduzido, no caso do Twitter limitadas a
140 caracteres.
Devido a esse fato e ao meio no qual as mensagens estão inseridas, é observáveis variações idiomáticas especificas deste
contexto.
Outro traço notável é que por ser uma plataforma aberta, o Twitter é composto de mensagens dos mais diversos domínios.
Esse fator dificulta a tarefa a ser executada quando comparado com trabalhos de um domínio específico, como por exemplo
a análise de sentimento para avaliação de filmes a partir de resenhas; ou na predição de flutuações do mercado financeiro
considerando artigos jornalísticos.
Além das características textuais, os \textit{tweets} também são compostos de atributos como: republicação ou citação de
mensagens, a presença de fotos e video, as conexões entre os usuários, local de envio, horário, número de curtidas etc.
Tais variáveis podem colaborar na realização de tarefas como a análise de sentimento, porém, este trabalho aborda apenas
as características textuais.

\section{Bases de Dados}

O trabalho de Go \textit{et at.}~\cite{go09} visa definir um sistema de classificação de sentimento sem a necessidade de
anotação manual dos dados para treinamento, utilizando a técnica de supervisão distante.
A não dependência de anotação dos dados e a facilidade de coleta dos \textit{tweets}, por meio de interface programável,
viabilizam a formação de uma grande base de treinamento.
Para a realização deste trabalho foram coletados cerca de 50 milhões de mensagens filtradas apenas pelo idiomas inglês,
estes \textit{tweets} compõe a base de treinamento.
Posteriormente, se aplica o método de classificação ruidosa de sentimento desenvolvido por Read~\cite{read05} como
especificado na seção~\ref{sec:distant_supervision}.
Nesse método, são definidos grupos de \textit{emoticons} positivos e negativos.
Mensagens que possuírem \textit{emoticons} pertencentes a algum destes grupos serão anotadas com as respectivas classes,
caso \textit{emoticons} de ambos os grupos estejam presentes em uma mesma mensagem esta será descartada.
Os \textit{emoticons} utilizados para anotação da base de dados foram removidos das mensagens para evitar introduzir
viés ao classificador.

A base de teste por sua vez é composta de \textit{tweets} anotados manualmente.
Essa base é formada pela coletânea de dados disponibilizados pela conferencia anual \textit{Semantic Evaluation}
(SemEval)~\cite{semeval17} entre os anos de 2013 a 2017.
Esta base é composta de cerca de 70 mil \textit{tweets} e processo de anotação destes dados foi feito manualmente
através da plataforma CrowdFlower, mais detalhes sobre a formação da base de dados são apresentados por Rosenthal
\textit{et al.}~\cite{rosenthal17}, organizadores do evento.
Novamente foram removidos os \textit{emoticons} presentes na anotação ruidosa.

\section{Desenvolvimento}

Na primeira etapa utilizou-se a base de dados utilizada por Go \textit{et al.}~\cite{go09} replicando as técnicas
abordadas em seu artigo visando validar os pré-processamentos e algoritmos aplicados.
Começou-se tokenizando os \textit{tweets}, durante esse processo removeram-se stopwords; links; referências a usuários;
e cada token foi transformado para forma minuscula.
Posteriormente, como entrada do algoritmo de aprendizado replicou-se apenas a representação dos tokens por unigrama,
essa escolha foi feita por ser a representação mais simples e por Go \textit{et at.}~\cite{go09} mostrar que há pouca
variação de resultado entre as diferentes representações.
Finalmente, foram aplicados os algoritmos de Naïve Bayes e SVM.
Dada a grande quantidade de dados e as limitações computacionais, o modelo SVM empregado foi treinado a partir do método
do gradiente como descrito por Suykens e Vandewalle~\cite{suykens99}.
Os hiperparâmetros utilizados por ambos os algoritmos foi selecionado a fim de otimizar a área sobre a curva ROC por
meio de validação cruzada.

Uma vez validados os resultados obtidos por Go \textit{et al.}~\cite{go09}, aplicou-se os mesmos processos na base de
dados elaborada por supervisão distante.
A comparação entre os resultados da etapa anterior e esta visam validar o processo de captação de dados.

Para utilização de modelos de redes convolucionais, obtém-se primeiro uma representação dos dados em um
\textit{embedding}.
Foram avaliados duas opções: a utilização de um \textit{embedding} Word2Vec pré-treinado pelo Google a partir de
notícias, como realizado por Kim~\cite{kim14} em seu trabalho; o treinamento de um \textit{embedding} Word2Vec a partir
da base de \textit{tweets} de treinamento.
Os parâmetros de treinamento do Word2Vec treinado a partir de \textit{tweets} foi baseados no modelo disponibilizado
pelo Google

Por fim, gerou-se um modelo de redes convolucionais aplicando como entrada a representação obtida pelo Word2Vec.
A escolha dos hiperparâmetros a serem testados foi baseada na submissão ao SemEval de Derius
\textit{et al.}~\cite{deriu16}, a qual obteve o melhor resultado da SemEval 2016.
Foram testados tanto modelos com aplicação de \textit{fine-tuning} no \textit{embedding} quanto modelos em que os pesos
do Word2Vec mantinham-se estáticos durante o treinamento.
A seleção dos hiperparâmetros foi determinada de maneira a maximizar a área sob a curva ROC.
Entretanto, neste caso não foi utilizada a validação cruzada visto o alto custo computacional de se treinar modelos de
\textit{deep learning}.
