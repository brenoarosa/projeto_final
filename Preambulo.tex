% Declaracao
\begin{center}
Declaração de Autoria e de Direitos
\end{center}

\vspace{0.5cm}

Eu, \emph{Breno Vieira Arosa} CPF \emph{131.187.117-95}, autor da monografia \emph{Aprendizado Semi-Supervisionado para Classificação de Polaridade de Tweets}, subscrevo para os devidos fins, as seguintes informações:\\
1. O autor declara que o trabalho apresentado na disciplina de Projeto de Graduação da Escola Politécnica da UFRJ é de sua autoria, sendo original em forma e conteúdo.\\
2. Excetuam-se do item 1. eventuais transcrições de texto, figuras, tabelas, conceitos e ideias, que identifiquem claramente a fonte original, explicitando as autorizações obtidas dos respectivos proprietários, quando necessárias.\\
3. O autor permite que a UFRJ, por um prazo indeterminado, efetue em qualquer mídia de divulgação, a publicação do trabalho acadêmico em sua totalidade, ou em parte. Essa autorização não envolve ônus de qualquer natureza à UFRJ, ou aos seus representantes.\\
4. O autor pode, excepcionalmente, encaminhar à Comissão de Projeto de Graduação, a não divulgação do material, por um prazo máximo de 01 (um) ano, improrrogável, a contar da data de defesa, desde que o pedido seja justificado, e solicitado antecipadamente, por escrito, à Congregação da Escola Politécnica.\\
5. O autor declara, ainda, ter a capacidade jurídica para a prática do presente ato, assim como ter conhecimento do teor da presente Declaração, estando ciente das sanções e punições legais, no que tange a cópia parcial, ou total, de obra intelectual, o que se configura como violação do direito autoral previsto no Código Penal Brasileiro no art.184 e art.299, bem como na Lei 9.610.\\
6. O autor é o único responsável pelo conteúdo apresentado nos trabalhos acadêmicos publicados, não cabendo à UFRJ, aos seus representantes,  ou ao(s) orientador(es), qualquer responsabilização/ indenização nesse sentido.\\
7. Por ser verdade, firmo a presente declaração.\\

      \vspace{0.5cm}
      \begin{flushright}
         \parbox{10cm}{
            \hrulefill

            \vspace{-.375cm}
            \centering{Breno Vieira Arosa}

            \vspace{0.1cm}
         }
      \end{flushright}

\pagebreak

% Copyright
\vspace{0.5cm}

UNIVERSIDADE FEDERAL DO RIO DE JANEIRO \\
Escola Politécnica - Departamento de Eletrônica e de Computação \\
Centro de Tecnologia, bloco H, sala H-217, Cidade Universitária \\
Rio de Janeiro - RJ      CEP 21949-900\\
\vspace{0.5cm}

Este exemplar é de propriedade da Universidade Federal do Rio de Janeiro, que poderá incluí-lo em base de dados, armazenar em computador, microfilmar ou adotar qualquer forma de arquivamento.

É permitida a menção, reprodução parcial ou integral e a transmissão entre bibliotecas deste trabalho, sem modificação de seu texto, em qualquer meio que esteja ou venha a ser fixado, para pesquisa acadêmica, comentários e citações, desde que sem finalidade comercial e que seja feita a referência bibliográfica completa.

Os conceitos expressos neste trabalho são de responsabilidade do(s) autor(es).

\pagebreak

% Dedicatória
\begin{center}
\textbf{DEDICATÓRIA}
\end{center}
\vspace{0.5cm}

\paragraph{}Opcional.

\pagebreak


% Agradecimento
\begin{center}
\textbf{AGRADECIMENTO}
\end{center}
\vspace{0.5cm}

\paragraph{}Sempre haverá. Se não estiver inspirado, aqui está uma sugestão: dedico este trabalho ao povo brasileiro que contribuiu de forma significativa à minha formação e estada nesta Universidade. Este projeto é uma pequena forma de retribuir o investimento e confiança em mim depositados.

\pagebreak

% Resumo
\begin{center}
\textbf{RESUMO}
\end{center}
\vspace{0.5cm}

O presente trabalho apresenta a elaboração de um classificador de \textit{deep learning} para análise de sentimento de
mensagens de redes sociais sem a necessidade de bases de dados anotadas manualmente.
Tendo em vista o crescimento de influência da mídias sociais, são propostos como objeto de estudos as mensagens da rede
social Twitter, as quais serão classificadas em positivas ou negativas.
Dado o custo de elaboração de bases de dados anotadas manualmente, principalmente se considerando a dependência de
grandes volumes de dados para o sucesso de técnicas de \textit{deep learning}, utiliza-se supervisão distante para
formação de base de dados de treinamento.
São abordados o uso de redes neurais convolucionais para classificação de texto e comparados seus resultados com
algoritmos de \textit{machine learning} tradicionalmente aplicadas no processamento de linguagem natural.

\vspace{0.5cm}

\noindent Palavras-Chave: análise de sentimento, machine learning, redes neurais convolucionais, processamento de
linguagem natural, supervisão distante.

\pagebreak

% Abstract
\begin{center}
\textbf{ABSTRACT}
\end{center}
\vspace{0.5cm}

The present work presents the elaboration of a deep learning classifier for sentiment analysis of messages of social
networks without the need of annotated datasets.
In view of the growing influence of social media, Twitter social network messages are proposed as object of study, which
will be classified in positive or negative.
Given the cost of developing manually annotated datasets, especially considering the dependence of large volumes
of data for the success of deep learning techniques, distant supervision is used to build a training dataset.
The use of convolutional networks for text classification is discussed and their results compared with machine learning
algorithms traditionally applied in natural language processing.

\vspace{0.5cm}

\noindent Keywords: sentiment analysis, machine learning, convolutional neural networks, natural language processing,
distant supervision

\pagebreak

% Siglas
\begin{center}
\textbf{SIGLAS}
\end{center}
\vspace{0.5cm}

\paragraph{}UFRJ - Universidade Federal do Rio de Janeiro
\paragraph{}EMQ - Erro Médio Quadrático
\paragraph{}SVM - \textit{Suport Vector Machine}


\pagebreak
