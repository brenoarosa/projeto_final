A massificação do uso das redes sociais gera uma crescente produção de dados.
Entretanto, as informações que estão sendo geradas são dispostas de forma não estruturada, dificultando sua extração.
Dado o volume de dados produzidos, torna-se cada vez mais custoso realizar esse processo manualmente.
Portanto, é fundamental o desenvolvimento de técnicas de processamento de linguagem natural capazes de auxiliar neste
procedimento.

O presente trabalho teve como objetivo o desenvolvimento de um método que gerasse classificadores de análise de sentimento
para redes sociais, sem uma dependência de anotação manual de bases de dados.
Classificadores formados por algoritmos de aprendizado de máquina vêm obtendo bons resultados na mineração de opinião.
Todavia, eles dependem de dados de treinamento, os quais têm produção custosa visto que sua criação depende da anotação
manual dos casos, dificultando, por exemplo, a reprodução destes classificadores para diferentes idiomas ou redes
sociais.

Com o desenvolvimento de classificadores cada vez mais complexos, como os englobados pelas técnicas de
\textit{Deep Learning}, cresce a dependência de grandes volumes de dados de treinamento.
O método da anotação ruidosa por supervisão distante, apesar de apresentar dificuldades decorrentes da seleção das
características correlacionada com as classes, se mostrou eficiente na formação de bases de treinamento.

Observou-se que, embora algoritmos de aprendizado de máquina tradicionalmente aplicadas ao processamento de linguagem
natural atinjam resultados positivos nesta tarefa, a utilização de técnicas de \textit{Deep Learning} é capaz elevar o
nível de desempenho obtido.

Portanto, foi desenvolvido um método de produção de classificadores eficazes e que não dependem de anotação manual de
dados de treinamento.
Isso se fez possível pela utilização de representações de texto por algoritmos de aprendizado de máquina não
supervisionados e por classificadores compostos por redes neurais convolucionais, treinadas a partir de dados anotados
por supervisão distante.

\section{Trabalhos Futuros}

Uma das limitações da supervisão distante para a análise de sentimento é sua incapacidade de formar dados de treinamento
para classes neutras.
Aponta-se como um trabalhos futuros o desenvolvimento de técnicas de anotação ruidosa capazes de lidar com esse problema.

Outro fator a ser estudado é o desempenho de outros algoritmos da família de técnicas de \textit{Deep Learning} para
realização de análise de sentimento.
A aplicação de técnicas como \textit{Long Short Term Memory} vem obtendo sucesso em outras tarefas de processamento de
linguagem natural e se mostra uma abordagem promissora.
