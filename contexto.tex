% Acertar titulo do capitulo

A internet apresenta um crescimento exponencial no mundo atual.
Estima-se que no ano de 2016 cerca de 66\% da população brasileira tenha acesso à rede \cite{social17}.
Ela é usada massivamente no dia a dia das pessoas, estando presente desde a realização de tarefas básicas e essenciais,
como cozinhar e se locomover numa cidade, até o preenchimento do tempo de lazer, com vídeos, notícias, mensagens
instantâneas e redes sociais.
% Falar sobre o uso da internet, ubuiquidade e etc

Através especialmente das redes sociais, a internet modificou a forma de interação entre as pessoas.
No Brasil, 58\% da população, ou seja, 120 milhões de pessoas, participam de pelo menos uma rede social, gastando em
média 220 minutos por dia~\cite{social17}.
Isso faz do Brasil o segundo país em tempo navegando em redes sociais, atrás apenas das Filipinas~\cite{social17}.
Isso não só cria uma nova instância de comunicação entre as pessoas, mas também abre um lugar para a expressão dos
sentimentos e opiniões de cada um.
Com isso, após o surgimento das redes sociais, a presença online deixou de ser uma comunicação de via única.
Agora, os leitores interagem com a fonte de informação recebida, seja ela uma notícia, atualização de amigos, etc.

Essa febre de compatilhamentos, em tempo real, de status, ideias, atividades e interesses faz com que
os usuários passem a gastar menos tempo navegando de forma independente e tendam a se guiar mais por recomendações de
seus contatos, que passam a servir como curadores de conteúdo ao qual um usuário terá acesso.

Nesse sentido, trabalhos que analisem o que se está dizendo nas redes sociais ganham importância comercial.
Para uma emissora de TV, por exemplo, é interessante saber o que está sendo dito sobre suas emissões, assim como para a
acessoria de um cantor, é importante descobrir se sua nova canção está agradando ou não o público.
O tema do presente trabalho visa a contribuir para essa questão.
Pretende-se analisar os sentimentos (positivo e negativo) presentes em \textit{tweets}.

A criação de conteúdo digital acompanha o crescimento da internet.
No ano de 2016, a cada sessenta segundos foram adicionadas 500 horas vídeo no Youtube, realizadas 3.8 milhões de buscas
no Google e publicados 450 mil \textit{tweets}~\cite{smartinsights}.

Pode-se observar que torna-se inviavel a realização de análise de sentimento manualmente sobre mídias com esse volume
de dados.
Portanto, a utilização de métodos de aprendizado de máquina é um modo de viabilizar a realização dessa tarefa.

Jim Yu, presidente da empresa de marketing digital BrightEdge, ressalta que os sentimentos expressados por clientes em
relação a uma marca é uma das principais métricas de \textit{branding}~\cite{marketingland}.
O colunista da Forbes, Daniel Newman, ressalta que a utilização desta e outras técnicas de big data vem revolucionando
a forma de se fazer marketing, permitindo uma maior personalização dos produtos e fortalecimento de relaçao com os
clientes.
Além de ser utilizada como métrica de relações comerciais, análises de sentimento de mídias sociais são aplicadas na
predição de mercados financeiros~\cite{bollen11}, no monitoramento de eventos como eleições e olimpiadas, entre outros.

\section{Análise de Sentimento}

\section{Twitter}
% falar do twitter, possivelmente dar exemplo de tweet, explicar hashtag pq aparece em stopwords
