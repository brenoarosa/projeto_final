A massificação do uso das redes sociais gera uma crescente produção de dados.
Entretanto, as informações que estão sendo geradas são dispostas de forma não estruturada, dificultando sua extração.
Dado o volume de dados produzidos, torna-se cada vez mais custoso realizar esse processo manualmente.
Portanto, é fundamental o desenvolvimento de técnicas de processamento de linguagem natural capazes de auxiliar neste
procedimento.

O presente trabalho teve como objetivo o desenvolvimento de um método que gerasse classificadores de análise de sentimento
para redes sociais, sem uma dependência de anotação manual de bases de dados.
Classificadores formados por algoritmos de aprendizado de máquina vêm obtendo bons resultados na mineração de opinião.
Todavia, eles dependem de dados de treinamento, os quais têm produção custosa visto que sua criação depende da anotação
manual dos casos, dificultando, por exemplo, a reprodução destes classificadores para diferentes idiomas ou redes
sociais.

Para abordar esse problema, foram estabelecidas três etapas.
A primeira etapa do desenvolvimento deste trabalho teve como objetivo replicar o artigo apresentado por Go
\textit{et al.}~\cite{go09} de maneira a validar os pré-processamentos e implementações de algoritmos observados.
Foi obtida acurácia de 83,3\% e 83,0\% pelos modelos, utilizando Naïve Bayes e \textit{Support Vector Machine} respectivamente.
Estes valores estão próximos aos apresentados por Go \textit{et al.}, que obteve 81,3\% e 82,2\% com os modelos de Naïve
Bayes e SVM, nessa ordem.
Portanto, foi possível validar os pré-processamentos e a implementação dos algoritmos.

O segundo estágio visou a avaliar a base de dados de anotação ruidosa formada para esse trabalho,
aplicando os mesmos modelos previamente validados.
Os resultados alcançados por modelos treinados nessa base de dados foram inferiores aos obtidos pelos modelos treinados
pela base de dados disponibilizada por Go \textit{et al.}
Concluiu-se que o processo de formação de base de dados por supervisão distante foi mais ruidoso do que o atingido por
Go \textit{et al.}
Entretanto, o método da anotação ruidosa por supervisão distante, apesar de apresentar dificuldades decorrentes da
seleção das características correlacionada com as classes, se mostrou eficiente na formação de bases de treinamento.

Os resultados obtidos na fase anterior serviram como base de comparação para terceira etapa.
Essa etapa teve como finalidade analisar o desempenho de modelos de \textit{Deep Learning} aplicados à
análise de sentimento de \textit{tweets} e compará-los a algoritmos de aprendizado de maquina tradicionalmente empregados
no processamento de linguagem natural.
O modelo de rede neural convolucional alcançou o melhor valor de AUC, obtendo 0,738 e superando Naïve Bayes e
\textit{Support Vector Machine}, que obtiveram 0,688 e 0,693 respectivamente.
Analisou-se então, que embora algoritmos de aprendizado de máquina tradicionalmente aplicadas ao processamento de linguagem
natural atinjam resultados positivos nesta tarefa, a utilização de técnicas de \textit{Deep Learning} é capaz elevar o
nível de desempenho obtido.
Observou-se também que é possível utilizar representações Word2Vec treinadas em notícias mesmo quando os objetos de análise
são mensagens de redes sociais.

Portanto, foi desenvolvido um método de produção de classificadores eficazes e que não dependam de anotação manual de
dados de treinamento.
Isso se fez possível pela utilização de representações de texto por algoritmos de aprendizado de máquina não
supervisionados e por classificadores compostos por redes neurais convolucionais, treinadas a partir de dados anotados
por supervisão distante.

\section{Trabalhos Futuros}

Uma das limitações da supervisão distante para a análise de sentimento é sua incapacidade de formar dados de treinamento
para classes neutras.
Aponta-se como trabalho futuro o desenvolvimento de técnicas de anotação ruidosa, ou novos métodos, capazes de lidar
com esse problema.

Outro fator a ser estudado é o desempenho na análise de sentimento de outros algoritmos da família de técnicas de
\textit{Deep Learning}, as quais vêm obtendo sucesso em outras tarefas de processamento de linguagem natural e se mostram promissoras.
Um exemplo dessas técnicas são as \textit{Long Short Term Memory}, modelos que englobam em sua composição componentes
de memória temporal.
Tais componentes são fundamentais para uma análise de línguas.
